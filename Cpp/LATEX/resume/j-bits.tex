\documentclass[paper=a4,fontsize=11pt,svgnames]{temp} % KOMA-article class

\newcommand{\name}{张梦磊}
%\newcommand{\job}{算法研发工程师} % 讯飞 大疆
\newcommand{\job}{游戏研发工程师} % ThoughtWorks
%\newcommand{\job}{算法工程师-图像图形}
%\newcommand{\job}{算法工程师-计算机视觉}
%\newcommand{\job}{算法研发工程师}
%\newcommand{\job}{人工智能工程师}
%\newcommand{\job}{算法实习生}
\newcommand{\email}{mlzhang.wd@foxmail.com}
\newcommand{\phone}{130-0637-7386}
\newcommand{\github}{https://github.com/zmlhome}
\newcommand{\blog}{https://zmlhome.github.io/}
\newcommand{\Address}{湖北省武汉市武汉大学电子信息学院}
\newcommand{\photo}{me.jpg}

\usepackage{fontspec}
%\setmainfont{Sauce Code Powerline Black} %{SourceCodePro-Bold}
\setmainfont{Source Code Pro}
\setsansfont{Source Code Pro}
\setmonofont{Source Code Pro}
\setCJKmainfont{微软雅黑}

\newcommand{\ftsize}{\zihao{5}}
\newcommand{\myfaEnvelope}{\textcolor{headings}{\faEnvelope}}
\newcommand{\myfaPhone}{\textcolor{headings}{\faPhone}}
\newcommand{\myGithub}{\textcolor{headings}{\faGithub}}
\newcommand{\myfaGlobe}{\textcolor{headings}{\faGlobe}}
\newcommand{\myfaHome}{\textcolor{headings}{\faHome}}

\newcommand{\MyJob}[1]{ % Job 
	\zihao{-2} \usefont{OT1}{phv}{b}{n} \hfill \textcolor{headings}{#1}
	\par \normalsize \normalfont}
\newcommand{\MyEmail}[1]{ % Job 
	\ftsize\myfaEnvelope\,  {#1}}
\newcommand{\MyPhone}[1]{ % Job 
	\ftsize\myfaPhone\,\,  {#1}}
\newcommand{\MyGithub}[1]{ % Job 
	\ftsize\myGithub\,\,  \href{#1}{#1}}
\newcommand{\MyBlog}[1]{ % Job 
	\ftsize\myfaGlobe\,\, \href{#1}{#1}}
\newcommand{\MyAddress}[1]{%
	\ftsize\myfaHome\,\,   {\parbox[t]{4cm}{#1}}}

\newcommand{\mtNodeWithoutFig}[4]{
	\begin{tikzpicture}
	\node[textNodeStyle={text width=#2}]
	(#1) at #3 {#4};
%	\figNodePos{f1}{#5}{#6}{right}{#1}{#7};
	\end{tikzpicture}
}
\newcommand{\ExperienceEntry}[5]{%
	\noindent%
	\hspace*{1em}%	                    
	\begin{minipage}{0.96\linewidth}  
		\noindent {\color{subheadings}\bfseries {#1}}, {\color{subheadings}\textit{#3}}  
		\hfill  {\color{headings}\fontsize{10pt}{12pt}#2}   
		\par\vspace*{.5em} 	
		\noindent\hangindent=2em\hangafter=0 \normalsize
		\hspace*{-1.3em}
		\mtNodeWithoutFig{t1}{.98\textwidth}{(0, 0)}{%
			#4 % Description
			\par
			\small\keyword #5
		}%
		\normalsize\par\vspace*{-.4em}         
	\end{minipage}         
}
							
\begin{document}

\newcommand{\namey}{-1.43}
\newcommand{\joby}{\namey - 1.1}
\newcommand{\photoy}{\namey - 1.17}
\newcommand{\emaily}{\namey + 0.23}
\newcommand{\phoney}{\emaily - 0.55}
\newcommand{\githuby}{\phoney - 0.55}
\newcommand{\blogy}{\githuby - 0.55}
\newcommand{\addressy}{\blogy - 0.8}


\begin{tikzpicture}[remember picture,overlay]
	\tNode{name}{([shift={(2.7, \namey)}]current page.north west)}{\MyName{\name}};
	\node[textNodeStyle,anchor=west] (job) at ([shift={(1.15, \joby)}]current page.north west) {\MyJob{\job}};
	\node(photo) at ([shift={(-2.7, \photoy)}]current page.north east)			
	{\includegraphics[width=2.2cm,height=3.2cm]{\photo}};
	\node[fit=(photo),draw=headings,rectangle,inner sep=-2pt,shift={(-.005, .01)}] {};
	\node(email) at ([shift={(-7.0, \emaily)}]current page.north east) {\MyEmail{\email}};
	\node(phone) at ([shift={(-7.9, \phoney)}]current page.north east) {\MyPhone{\phone}};
	\node(github) at ([shift={(-6.75, \githuby)}]current page.north east) {\MyGithub{\github}};
	\node(blog) at ([shift={(-6.9, \blogy)}]current page.north east) {\MyBlog{\blog}};
%	\node(address) at ([shift={(-7.53, -3.85)}]current page.north east) {\MyAddress{\Address}}; %% without parbox
	\node(address) at ([shift={(-7.09, \addressy)}]current page.north east) {\MyAddress{\Address}};
\end{tikzpicture}

\vspace*{2.2cm}

\NewPart{教育背景}{}
\noindent


\EducationEntry{硕士 / 武汉大学 }{2016/09 -- 2019/06}{%
\begin{itemize}
\item 学院:电子信息学院 \hspace*{.8cm} 专业:信号与信息处理 \hspace*{.6cm} 工学硕士\hspace*{.6cm} 平均分:88.68 (GPA:3.66/4.0)
\item 研究方向:图像去噪与超分辨率,深度学习
\end{itemize}%
}

\sepspace

\EducationEntry{本科 / 武汉大学 }{2012/09 -- 2016/06}{%
\begin{itemize}
\item 学院:电子信息学院 \hspace*{.90cm}专业:通信工程  \hspace*{1.73cm} 工学学士\hspace*{.6cm} 平均分:84.08 (GPA:3.32/4.0)
\end{itemize}%
}

\NewPart{项目经验}{}
\noindent


\projectEntry{基于 DCNN 的图像去噪算法}{2017/07 -- 2018/03}{第一作者}{%
构建深度卷积神经网络对带噪图像进行残差学习,算法使用 PyTorch 深度学习框架\\ 实现,利用残差网络、ReLU 等技术加速算法收敛并提升去噪效果,在标准测试图\\ 像 Set12 上的平均 PSNR 值比 DnCNN 算法高出 0.15 $\sim$ 0.26dB 左右。预计投递\\  \textit{Signal Processing}(SCI 检索,IF:3.47),  关于该算法的报告可查看:\href{https://github.com/zmlhome/Amaze/blob/master/Layer/zml-annual-meeting-20180125.pdf}{2018 年会报告}。%掌握网络构建、\textbf{反向传播机制}、常用\textbf{优化算法},
}{\textbf{PyTorch}\quad \textbf{深度学习}\quad\textbf{残差网络}\quad 图像去噪}{network.pdf}{3.8cm}{(-1.85, .4)}

\sepspace

\projectEntry{人体消化道病变区域检测}{2018/03 -- 2018/04}{导师项目,第一负责人}{
	利用深度学习相关技术对人体消化道内的病变区域进行检测,模型基于 LeNet, 用\\ 于判断消化道内镜样图中的任意区域是否发生病变,并将病变区域标识出来。算法\\ 使用 Keras 实现,测试准确率达 92.4\%。
}{\textbf{Keras}\quad\textbf{LeNet}}{lesion.pdf}{3.9cm}{(-1.95, 0)}

\sepspace

%\projectEntry{源码阅读}{2017/10 -- 2018/01}{个人项目}{
%	%\textbullet\hspace*{.3em}
%	针对 PyTorch v0.1.1 源码进行分析,主要探究卷积层、ReLU 层等反向传播的底层实现,以及 PyTorch 的自动求导机制,提升对 PyTorch 框架的整体认识:\href{https://github.com/zmlhome/DownMak/tree/master/Python/pytorch}{PyTorch 源码详解}。
%%	\textbullet\hspace*{.3em}
%%	\href{https://www.kancloud.cn/ieric_1993/tiny_dnn}{看云文档:TinyDnn 源码阅读}\\
%%	\quad 针对 TinyDnn v0.0.1 源码进行分析,重点理解网络的构建以及反向传播机制。
%}{\textbf{PyTorch}\quad\textbf{Caffe}\quad\textbf{反向传播}\quad\textbf{自动求导}}{autograd.pdf}{1.6cm}{(-.6, -.2)}
%\vspace*{-.3cm}
%\sepspace

\projectEntry{同伦方法在图像稀疏去噪中的应用}{2016/10 -- 2017/06}{第一作者}{在稀疏表示和字典学习理论的基础上,本文算法利用同伦方法学习字典并结合稀疏去\\ 噪模型实现对图像的去噪。在与 K-SVD 算法的比较实验中,收敛速度大致是 K-SVD \\ 算法的 1.4 $\sim$ 2 倍,并获得更高的 PSNR,充分展示了同伦方法收敛速度快以及对信\\ 号的恢复准确度高等优势。 成果发表在\href{http://www.signal.org.cn/CN/volumn/current.shtml\#89}{《信号处理》2018 Vol.34(1): 89-97}。}{\textbf{Matlab}\quad 同伦方法\quad 字典学习\quad 稀疏表示}{denoise.pdf}{3.4cm}{(-1.4, 0)}

\NewPart{实习经历}{}
\noindent


\ExperienceEntry{京东集团 \quad 广告质量部}{2018/05 -- 2018/07}{算法工程师}{%
	前期主要从事素材生成方面的工作,采用泊松图像编辑算法实现背景与装饰图像的无缝融合,生成大量的广告素材;后期接触推荐系统,基于 PyTorch 深度学习框架实现 YouTube 推荐系统模型,并用于京东的用户数据,实现对用户的商品推荐。在已知 60 条用户过往历史的情况下,预测用户最近一次的浏览情况,top600 准确率达 70.2\%。
}{\textbf{素材生成}\quad \textbf{泊松图像编辑}\quad \textbf{推荐系统}}



\NewPart{外语与技能}{}
\noindent
\vspace*{-.9cm}
\hspace*{0cm}
%\begin{minipage}{1em}
%\rule{1em}{0pt}
%\end{minipage}
\begin{minipage}{.96\textwidth}
\begin{itemize}
	\item 英语水平:CET4 (成绩: \textbf{607}) \hspace*{1cm} CET6 (成绩: \textbf{551})
	%\item 编程语言与环境: Python > C++ > C;常在 Linux 下开发,编程环境为 vim + tmux + ctags;
	\item 深度学习框架:了解 \textbf{PyTorch},\textbf{Caffe} 等常用深度学习框架,掌握网络构建、常用优化算法,熟悉深度学习常用的网络结构,如 \textbf{ResNet},\textbf{VGG}等;熟悉目标检测算法 \textbf{Faster-RCNN}。%掌握网络构建、\textbf{反向传播机制}、常用\textbf{优化算法},
	\item 理论知识:扎实的计算机基础,了解\textbf{操作系统}、常用\textbf{数据结构},掌握 C++ 面向对象程序设计思想,掌握 C++ 对象模型的内存布局、虚函数、多态的实现机制,熟悉 STL 中各种容器和算法的使用。
	\item 排版系统:熟悉 \LaTeX{} 排版系统,有较强的\textbf{文档撰写能力} (作品:
	\href{http://www.latexstudio.net/archives/4200}{2015年全国大学生数学建模LaTeX模板})
\end{itemize}
\end{minipage}

%%% References
%%% ------------------------------------------------------------

\end{document}
