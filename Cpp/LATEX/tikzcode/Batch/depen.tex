\documentclass{article}
\usepackage{ctex}
\usepackage{tikz}
\usepackage{amsmath}
\usepackage{amssymb}
\pagecolor{pink}
\usetikzlibrary{arrows, positioning,calc}
\usepackage[active, tightpage]{preview}
\PreviewEnvironment{tikzpicture}
\definecolor{arrBlue}{HTML}{015EDF}
\newcommand{\arrcolor}{arrBlue}
\newcommand{\arrlinewidth}{1pt}
\tikzset{
	% 箭头和线条的样式, 用于 \draw
	arrStyle/.style = {->, >=stealth,
	line width=\arrlinewidth,#1},
	arrStyle/.default = {\arrcolor},
	lineStyle/.style = {line width=\arrlinewidth,
	\arrcolor},
}

\makeatletter
\newcommand{\nodefontsize}{\@setfontsize\codesize{2.5}{3.4}}
\makeatother 


\tikzset{%
	kernelStyle/.style={draw, circle,
	inner sep=0pt, minimum size=1.2em,
	outer sep=1pt, line width=1pt, #1, fill=#1,
	align=center,text=white,font=\nodefontsize},
	kernelStyle/.default={orange}
} %

% \kernel{< 颜色 >}{< 名字 >}{< 位置 >};
% \kernelPos{< 颜色 >}{< 名字 >}{< 相对位置 >}{< 基准名字 >}{< 相对基准的偏移 >};
\newcommand{\kernel}[3]{\node[kernelStyle=#1] (#2) at #3 {}}
\newcommand{\kernelPos}[5]{\node[kernelStyle=#1,
	#3=of #4, #3=0pt, shift={#5}] (#2) {}}
% 可以在神经元内添加文字, 前缀用 t 表示 text
\newcommand{\tkernel}[4]{\node[kernelStyle=#1] (#2) at #3 {#4}}
\newcommand{\tkernelPos}[6]{\node[kernelStyle=#1,
	#3=of #4, #3=0pt, shift={#5}] (#2) {#6}}

\tikzset{%
	textNodeStyle/.style={align=center,
	inner sep=0pt, %minimum size=2em,
	text width=.4cm,
	font=\nodefontsize,
	outer sep=1pt, #1},
	textNodeStyle/.default={},
	boxStyle/.style={line width=1pt,%
	rounded corners=3pt,
	}% 还可以加颜色
}

\newcommand{\tNode}[3]{\node[textNodeStyle] (#1) at #2 {#3}}
\newcommand{\tNodePos}[5]{\node[textNodeStyle,
#2=of #3, #2=0pt, shift={#4}] (#1) {#5}}


\tikzset{
	anodeStyle/.style={
		circle,fill=\arrcolor,inner sep=0pt,outer sep=0pt,minimum size=3pt %
	}
}
\newcommand{\tarrowNode}[5]{\draw[arrStyle] #1 -- #2 node (anode) [midway, anodeStyle] {};
\tNodePos{tt}{#3}{anode}{#4}{#5}}

\usepackage{minted}

\makeatletter
\newcommand{\codesize}{\@setfontsize\codesize{4.9}{8}}
\newcommand{\tinycodesize}{\@setfontsize\codesize{6.7}{9.6}} 
\makeatother 

\newminted[python]{python}{fontsize=\codesize,style=monokai, escapeinside=||, baselinestretch=.7}
\newmintinline[pyinline]{python}{}

\definecolor{convBlue}{HTML}{008DF4}
\definecolor{convDark}{HTML}{002947}
\definecolor{convGray}{HTML}{77AAFF}
\definecolor{convGreen}{HTML}{35CD74}
\definecolor{convPurple}{HTML}{CE357A}
\newcommand{\dis}{.3cm}
\begin{document}
\begin{tikzpicture}
\coordinate (origin) at (0, 0);
\tkernel{orange}{linear}{(origin)}{Linear};
\tkernelPos{convGreen}{leaf}{below}{linear}{(0, -2*\dis)}{Leaf};

\tarrowNode{(leaf)}{(linear)}{right}{(-.1, 0)}{};
\node[textNodeStyle,
left=of anode, align=left,text width=1.3cm,right=0pt, shift={(-1.3, -.1)}] (arg) {该 variable 相对 Leaf 为 output, 但相对 Linear 为 input};
\draw[arrStyle] (linear) -- ++(0, 3*\dis) node[midway, anodeStyle] (out) {};
\tNodePos{output}{right}{out}{(0, 0)}{variable};
\node[textNodeStyle,
right=of linear, text width=1cm,right=0pt, shift={(-.1, 0)}] (creator) {variable.creator};
\tNodePos{grad}{right}{linear}{(-.2, .35)}{\color{red}grad};
\draw[arrStyle, convPurple, densely dashed] (linear) edge[out=-45, in=45] (leaf) node[midway,right,font=\nodefontsize,shift={(.3, -.5)}] {Linear 是 Leaf 的依赖项};


%% text
%\tNodePos{fn}{right}{linear1}{(.3, .65)}{\color{red}fn};
%\draw[lineStyle, densely dashed, orange] (linear1) -- (fn);

\end{tikzpicture}
%\setlength{\PreviewBorder}{.03cm}
\end{document}