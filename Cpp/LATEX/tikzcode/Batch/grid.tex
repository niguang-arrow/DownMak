\documentclass{article}
\usepackage{ctex}
\usepackage{tikz}
\usepackage{amsmath}
\usepackage{amssymb}
\pagecolor{pink}
\usetikzlibrary{arrows, positioning,calc}
\usepackage[active, tightpage]{preview}
\PreviewEnvironment{tikzpicture}
%\setlength{\PreviewBorder}{1cm}
\definecolor{arrBlue}{HTML}{015EDF}
\newcommand{\arrcolor}{arrBlue}
\newcommand{\arrlinewidth}{1pt}
\tikzset{
	% 箭头和线条的样式, 用于 \draw
	arrStyle/.style = {->, >=stealth,
	line width=\arrlinewidth,#1},
	arrStyle/.default = {\arrcolor},
	lineStyle/.style = {line width=\arrlinewidth,
	\arrcolor},
}

\tikzset{%
	kernelStyle/.style={draw, rectangle,
	inner sep=0pt, minimum size=4em,
	outer sep=1pt, line width=1pt, #1, fill=#1},
	kernelStyle/.default={orange}
} %

% \kernel{< 颜色 >}{< 名字 >}{< 位置 >};
% \kernelPos{< 颜色 >}{< 名字 >}{< 相对位置 >}{< 基准名字 >}{< 相对基准的偏移 >};
\newcommand{\kernel}[3]{\node[kernelStyle=#1] (#2) at #3 {}}
\newcommand{\kernelPos}[5]{\node[kernelStyle=#1,
	#3=of #4, #3=0pt, shift={#5}] (#2) {}}
% 可以在神经元内添加文字, 前缀用 t 表示 text
\newcommand{\tkernel}[4]{\node[kernelStyle=#1] (#2) at #3 {#4}}
\newcommand{\tkernelPos}[6]{\node[kernelStyle=#1,
	#3=of #4, #3=0pt, shift={#5}] (#2) {#6}}

\tikzset{%
	textNodeStyle/.style={align=center,
	inner sep=0pt, minimum size=2em,
	text width=2.8cm,
	outer sep=1pt, #1},
	textNodeStyle/.default={},
	boxStyle/.style={line width=1pt,%
	rounded corners=3pt,
	}% 还可以加颜色
}

\newcommand{\tNode}[3]{\node[textNodeStyle] (#1) at #2 {#3}}
\newcommand{\tNodePos}[5]{\node[textNodeStyle,
#2=of #3, #2=0pt, shift={#4}] (#1) {#5}}

% \tarrowNode{(start pos)}{(end pos)}{relative position}{<content>};
\newcommand{\tarrowNode}[4]{\draw[arrStyle] #1 -- #2 node[midway, #3] {#4}}

\usepackage{minted}

\makeatletter
\newcommand{\codesize}{\@setfontsize\codesize{4.9}{8}}
\newcommand{\tinycodesize}{\@setfontsize\codesize{3.7}{6.6}} 
\makeatother 

\newminted[python]{python}{fontsize=\codesize,style=monokai, escapeinside=||, baselinestretch=.7}
\newmintinline[pyinline]{python}{}

% https://tex.stackexchange.com/questions/128064/how-to-overbrace-underbrace-in-text-mode-using-tikz
\usetikzlibrary{decorations.pathreplacing}

\tikzstyle{overbrace text style}=[font=\tiny, above, pos=.5, yshift=3mm]
\tikzstyle{overbrace style}=[decorate,decoration={brace,raise=2mm,amplitude=3pt}]
\tikzstyle{underbrace style}=[decorate,decoration={brace,raise=2mm,amplitude=3pt,mirror},color=gray]
\tikzstyle{underbrace text style}=[font=\tiny, below, pos=.5, yshift=-3mm]

\begin{document}
\begin{tikzpicture}
\coordinate (origin) at (0, 0);
\node[text width=4cm] at (origin) {\begin{tikzpicture}
% 网格
\draw[step=.2cm] (origin) grid ++(4.2cm, 4.2cm);
% orange image
\draw[orange,fill=orange] ([shift={(.6cm, .6cm)}]origin) rectangle ++(3cm, 3cm);
\tNode{t1}{([shift={(2cm, 2cm)}]origin)}{\textcolor{white}{image}};
% 指示什么是 padding
\path[arrStyle] ([shift={(2.54cm, 3.84cm)}]origin) edge[out=100, in=180] ++(.5, .5) coordinate (A);
\node[\arrcolor,font=\codesize] at ([shift={(2.6, .6)}]A) {网格为 padding};
% 3 个被 dilated 的 kernel
\draw[lineStyle] ([shift={(0, 4.2)}]origin) rectangle ++(1cm, -1cm);
\draw[lineStyle,green] ([shift={(.4, 4.2)}]origin) rectangle ++(1cm, -1cm);
\draw[lineStyle,purple] ([shift={(.8, 4.2)}]origin) rectangle ++(1cm, -1cm);
% 指示 w_stride 的大小
\draw [overbrace style] ([shift={(0, 4.1)}]origin) -- ([shift={(.4, 4.1)}]origin) node [overbrace text style,shift={(1.8cm, 0)}] {\tinycodesize$w_{stride}$};
% 指示 w_pad 的大小
\draw [underbrace style] ([shift={(0, 0.1)}]origin) -- ([shift={(.6, 0.1)}]origin) node [underbrace text style,shift={(1.8cm, 0)}] {pad};
% 红点1 (h_im, w_im)
\path[fill=red]([shift={(1.8, 3.6)}]origin) circle (.05cm) coordinate (point);
\path[arrStyle] ([shift={(.05, -0.05)}]point) edge[out=-45, in=180] ([shift={(.8, -.5)}]point);
\tNode{t2}{([shift={(1.2, -.5)}]point)}{\tinycodesize\textcolor{white}{$(h_{im}, w_{im})$}};
% 红点2 (h_col, w_col)
\path[fill=red]([shift={(.8, 4.2)}]origin) circle (.05cm) coordinate (point1);
\path[arrStyle] ([shift={(-.05, 0.05)}]point1) edge[out=125, in=180] ([shift={(0.2, 1.5)}]point);
\tNode{t3}{([shift={(0.6, 1.5)}]point)}{\tinycodesize\textcolor{white}{$(h_{col}, w_{col})$}};
% 根据 w_offset 找到在 dilated kernel 上的位置, 即 w_offset * w_dilation
\draw [overbrace style] ([shift={(0.8, 4.1)}]origin) -- ([shift={(1.6, 4.1)}]origin) node [overbrace text style,shift={(1.8cm, 0)}] {\tinycodesize $w_{offset}\times w_{dilation}$};
% 最后的公式
\node[textNodeStyle, font=\tinycodesize,text width=2.2cm, shift={(-.3, .1)}] (t1) at ([shift={(1.1, -.8)}]origin) {
\begin{python}
int w_im = w_col * stride_w - pad_w + w_offset * dilation_w;
int h_im = h_col * stride_h - pad_h + h_offset * dilation_h;
\end{python}
};
\end{tikzpicture}};
% 特殊作用, 用于占位, 让右边的内容显示出来.
\node[minimum size=1cm] at (2.6cm, 0) {};
\end{tikzpicture}
\end{document}