\begin{dispExample*}{%
	 sidebyside,
	 lefthand ratio=0.7,
	 halign lower=right}
	
\definecolor{arrBlue}{HTML}{015EDF}
\newcommand{\arrcolor}{arrBlue}
\newcommand{\arrlinewidth}{6pt}

\tikzset{
arrStyle/.style = {->, >=stealth, 
	line width=\arrlinewidth,#1},
arrStyle/.default = {\arrcolor}
}	
	
\begin{tikzpicture}
\newcommand{\radius}{2}
\coordinate (origin) at (0, 0);
\draw[arrStyle] (origin) -- ++(\radius, 0);

% 连接 (A) -| -- (B)
% 先得到 (A) 和 (B) 直线连接的中间点 (C)
% 由于使用了 (C|-B), 所以还需要考虑 (C-|B) 的情况
\node (A) at ([shift={(0, -1)}]origin) {A};
\node (B) at ([shift={(3, -2)}]origin) {B};
\coordinate (C) at ($(A)!.5!(B)$) node[right=of C,right=4pt] {C};
\draw[arrStyle] (A)
	-| (C|-B) -- (B);

% 另一种情况
\node (A1) at ([shift={(3, -3)}]origin) {A};
\node (B1) at ([shift={(.5, -5)}]origin) {B};
\coordinate (C1) at ($(A1)!.5!(B1)$) node[above=of C1,above=4pt] {C};
\draw[arrStyle] (A1) |- (C1-|B1) -- (B1);

% 弯曲
\node (A2) at ([shift={(0, -7)}]origin) {A};
\node (B2) at ([shift={(3, -5)}]origin) {B};
\path[arrStyle] (A2.east) edge[out=0, in=180] (B2.west);

% rounded corners
\node (A3) at ([shift={(0, -9)}]origin) {A};
\node (B3) at ([shift={(3, -7)}]origin) {B};
\coordinate (C3) at ($(A3)!.5!(B3)$) node[right=of C3,right=4pt] {C};
\draw[arrStyle, rounded corners=5pt] (A3)
-| (C3|-B3) -- (B3);

% 换种风格
\draw[->, line width=4pt] ([shift={(0, -10)}]origin) -- ++(\radius, 0);
\end{tikzpicture}
\end{dispExample*}