\subsection{本文档中的写作}

本文档用了方框来展示 \tikzlogo 代码以及相应的结果. 要达到这样的效果, 需要使用如下的代码:

\begin{minted}{latex}
% 在导言区
\usepackage{tcolorbox}
\tcbuselibrary{documentation,minted}
\tcbset{listing engine=minted}
\tcbset{%
	docexample/.style={colframe=gray!40!white,colback=ExampleBack,
		before skip=\medskipamount,after skip=\medskipamount,
		fontlower=\footnotesize, 
		documentation minted options={fontsize=\zihao{-5}},}%
}
\end{minted}

之后写作的时候, 只需要将相应的 \tikzlogo 代码放在 \lxcode|dispExample*| 环境中, 使用带 \lxcode|*| 的环境则必须加上环境选项, 而 \lxcode|dispExample| 则不需要选项. 但我只用 \lxcode|dispExample*| 环境, 以便控制显示效果. 例子如下:\\[-.3cm]

\noindent 左右排列:

\begin{minted}{latex}
\begin{dispExample*}{%
		sidebyside,
		lefthand ratio=0.7,
		halign lower=right}
	
\definecolor{arrBlue}{HTML}{015EDF}
\newcommand{\arrcolor}{arrBlue}
\newcommand{\arrlinewidth}{6pt}

\tikzset{
	arrStyle/.style = {->, >=stealth, 
		line width=\arrlinewidth,#1},
	arrStyle/.default = {\arrcolor}
}	

\begin{tikzpicture}
\newcommand{\radius}{2}
\coordinate (origin) at (0, 0);
\draw[arrStyle] (origin) -- ++(\radius, 0);
\end{tikzpicture}
\end{dispExample*}
\end{minted}


\noindent 上下排列:

\begin{minted}{latex}
% 在 dispExample* 环境中使用 tikz 代码即可.
\begin{dispExample*}{%
		halign lower=center}
\tikzstyle{layer1} = [inner sep=0pt,minimum size=1em, outer sep=1pt]
\begin{tikzpicture}
% 输入节点
\node[layer1] (x1) {$x_1$};
\node[layer1,below=of x1, below=.5cm] (x2) {$x_2$};
\end{tikzpicture}
\end{dispExample*}
\end{minted}