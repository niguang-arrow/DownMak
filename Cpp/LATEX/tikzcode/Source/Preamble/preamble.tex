\subsection{图像较简单的情况}

\begin{minted}{latex}
	
\documentclass{article}

\usepackage{ctex}
\usepackage{tikz}
\usetikzlibrary{arrows, positioning,calc}

\usepackage[active, tightpage]{preview}
\PreviewEnvironment{tikzpicture}

\definecolor{arrBlue}{HTML}{015EDF}
\newcommand{\arrcolor}{arrBlue}
\newcommand{\arrlinewidth}{6pt}

\tikzset{
	% 箭头和线条的样式, 用于 \draw
	arrStyle/.style = {->, >=stealth, 
		line width=\arrlinewidth,#1},
	arrStyle/.default = {\arrcolor},
	lineStyle/.style = {line width=\arrlinewidth, 
		\arrcolor},
}

\begin{document}
...
\end{document}
\end{minted}

\subsection{画深度学习方面的图}

\begin{minted}{latex}
	
\documentclass{article}

\usepackage{ctex}
\usepackage{tikz}
\usetikzlibrary{arrows, positioning,calc}

\usepackage[active, tightpage]{preview}
\PreviewEnvironment{tikzpicture}

\pgfdeclarelayer{secbackground}
\pgfdeclarelayer{background}
\pgfdeclarelayer{foreground}
\pgfsetlayers{secbackground,background,main,foreground}

\definecolor{arrBlue}{HTML}{015EDF}
\newcommand{\arrcolor}{arrBlue}
\newcommand{\arrlinewidth}{6pt}

\tikzset{
	% 箭头和线条的样式, 用于 \draw
	arrStyle/.style = {->, >=stealth, 
		line width=\arrlinewidth,#1},
	arrStyle/.default = {\arrcolor},
	lineStyle/.style = {line width=\arrlinewidth, \arrcolor},
	nodeStyle/.style = {inner sep=0pt,
		line width=\arrlinewidth,
		color=\arrcolor!50!red,
		minimum size=2cm},
}

\makeatletter
\tikzset{%
	layerStyle/.pic = {
		\tikzset{
			/layer/.cd,
			#1,
		}
		\coordinate (layer@O) at (0, 0, 0);
		\path[fill=\layer@color!75!white] (layer@O) -- 
			++(-\layer@depth,0,0) coordinate (layer@A) --
			++(0,-\layer@height,0) coordinate (layer@B) --
			++(\layer@depth,0,0) coordinate (layer@C) 
			-- cycle;
		\coordinate (layer@Center) at ($(layer@O)!.5!(layer@B)$);
		\node[font=\color{\layer@textcolor}
			\bfseries\zihao{\layer@fontsize},rotate=90] 
			at (layer@Center) {\layer@content};
		\path[fill=\layer@color] (layer@O) -- 
			++(0,0,-\layer@width) coordinate (layer@D) --
			++(0,-\layer@height,0) coordinate (layer@E) -- 
			(layer@C) -- cycle;
		\path[fill=\layer@color!55!white] (layer@O) -- (layer@A) -- 		
			++(0,0,-\layer@width) coordinate (layer@F)  -- 
			(layer@D) -- cycle;
	},
	/layer/.cd,
	depth/.store in=\layer@depth,
	height/.store in=\layer@height,
	width/.store in=\layer@width,
	angle/.store in=\layer@angle,
	color/.store in=\layer@color,
	content/.store in=\layer@content,
	textcolor/.store in=\layer@textcolor,
	fontsize/.store in=\layer@fontsize,
	depth=.6pt,
	height=2.2pt,
	width=1pt,
	angle=60,
	color=black!20!orange,
	content=,
	textcolor=black,
	fontsize=5,
}

\tikzset{
	contentNodeStyle/.pic = {
		\tikzset{
			/contentNode/.cd,
			#1	
		},
		\node[shape=\content@shape, 
		inner sep=\content@innersep, 
		fill=\content@fill, 
		font=\color{\content@textcolor}
			\bfseries\zihao{\content@fontsize}] 
		at (0 , 0)  {\content@content};
	},
	/contentNode/.cd,
	fill/.store in=\content@fill,
	textcolor/.store in=\content@textcolor,
	content/.store in=\content@content,
	inner sep/.store in=\content@innersep,
	shape/.store in=\content@shape,
	fontsize/.store in=\content@fontsize,
	fill=blue,
	textcolor=black,
	content=,
	inner sep=2pt,
	shape=circle,
	fontsize=5,
}

\makeatother

%% 主要提供下面这 3 条命令
\newcommand{\contentNode}[3]{
	\node (#1) at #2 
		{\tikz\pic at (0, 0) {contentNodeStyle={#3}};}
}
\newcommand{\layerNode}[3]{
	\node (#1) at #2 
		{\tikz\pic at (0, 0) {layerStyle={#3}};}
}
\newcommand{\layerNodePos}[5]{
	\node (#1)  at ([shift={#4}]#3.#2) 
		{\tikz\pic at (0, 0) {layerStyle={#5}};}
}

\begin{document}
...
\end{document}
\end{minted}

对于这种情况, 稍微复杂一些, 主要提供最后的 3 条命令, 使用方法可以参考\textbf{深度学习}这一节.