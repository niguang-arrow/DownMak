\subsection{流程图}

\begin{dispExample*}{%
		sidebyside,
		lefthand ratio=0.7,
		halign lower=center}
% 这里给出的是流程图的 shape
% 需要使用 \usetikzlibrary{arrows,shapes}
% 例子中用到 \vv 向量命令和 \bm, 需要 \usepackage{esvect}
% 和 \usepackage{bm}
\tikzstyle{start}=[ellipse,draw,text=red]
\tikzstyle{end}=[ellipse,draw,text=red]
\tikzstyle{initial}=[rectangle,draw,rounded corners=4pt,
fill=blue!25]
% 指令
\tikzstyle{instruct}=[rectangle,draw,fill=yellow!50]
% 判断
\tikzstyle{test}=[diamond, aspect=2.5,thick,
draw=blue,fill=yellow!50,text=blue]
% 箭头
\tikzstyle{suite}=[->,>=stealth',thick,rounded corners=4pt]

\begin{tikzpicture}
\coordinate (origin) at (0, 0);
\newcommand{\dis}{1cm}
\node[start](start) at (origin) {Input: A, y};
\node[initial, below=of start, below=\dis](initial)  	
	{\shortstack{$x^\ast\leftarrow\vv{\bm{0}}$,\\
	 $\tau\leftarrow\max_i\vert\Phi^Ty\vert$}};
\node[test, below=of initial, below=\dis] (test) 
	 {$\tau > \varepsilon ?$};
\node[instruct,below=of test, below=\dis](instruct)  
	{$\partial x\leftarrow (\Phi_\Gamma^T\Phi_\Gamma)^{-1}z$};
\node[end, below=of instruct, below=\dis] (end) 
	{Output: $x^\ast$};
\draw[suite] (start) -- (initial);
\draw[suite] (initial) -- (test);
\draw[suite] (test) -- (instruct) node[midway,fill=ExampleBack]{Yes};
\draw[suite] (instruct) -- (end);
\coordinate (B) at ($(instruct)!.5!(end)$);
\coordinate (C) at ([shift={(2.6, 0)}]$(test)!.5!(B)$);
\draw[suite] (test)  -| (C|-B) -- (B) ;
\node[right=of test, right=3pt, fill=ExampleBack]{No};
\end{tikzpicture}
\end{dispExample*}

\subsection{神经元}

\begin{dispExample*}{%
		sidebyside,
		lefthand ratio=0.7,
		halign lower=center}
\tikzset{%
	neuralStyle/.style={draw, circle,
		inner sep=0pt, minimum size=2em,
		outer sep=1pt, line width=1pt, #1},
	neuralStyle/.default={orange}
}
% 不要忘了最后的分号;
% \neural{<颜色>}{<名字>}{<位置>};
% \neuralPos{<颜色>}{<名字>}{<相对位置>}{<基准名字>}{<相对基准的偏移>};
\newcommand{\neural}[3]{\node[neuralStyle=#1] (#2) at #3 {}}
\newcommand{\neuralPos}[5]{\node[neuralStyle=#1,
		#3=of #4, #3=0pt, shift={#5}] (#2) {}}
% 可以在神经元内添加文字, 前缀用 t 表示 text
\newcommand{\tneural}[4]{\node[neuralStyle=#1] (#2) at #3 {#4}}
\newcommand{\tneuralPos}[6]{\node[neuralStyle=#1,
	#3=of #4, #3=0pt, shift={#5}] (#2) {#6}}

\begin{tikzpicture}
\coordinate (origin) at (0, 0);
\neural{orange}{x}{(origin)};
\neuralPos{blue}{x1}{below}{x}{(0, -.3)};
\neuralPos{green}{x2}{below}{x1}{(0, -.3)};

% 添加文本
\tneural{orange}{x}{([shift={(1, 0)}]origin)}{$a_1^{(2)}$};
\tneuralPos{blue}{x1}{below}{x}{(0, -.3)}{$x_1$};
\tneuralPos{green}{x2}{below}{x1}{(0, -.3)}{$O_1$};
\end{tikzpicture}
\end{dispExample*}

\subsection{文本框}

\begin{dispExample*}{%
		sidebyside,
		lefthand ratio=0.7,
		halign lower=center}
\tikzset{%
	textNodeStyle/.style={align=center,
		inner sep=0pt, minimum size=2em,
		outer sep=1pt, #1},
	textNodeStyle/.default={},
	boxStyle/.style={line width=1pt,%
			rounded corners=3pt,
		}% 还可以加颜色
}

\newcommand{\tNode}[3]{\node[textNodeStyle] (#1) at #2 {#3}}
\newcommand{\tNodePos}[5]{\node[textNodeStyle,
	#2=of #3, #2=0pt, shift={#4}] (#1) {#5}}

% 多行文本
\newcommand{\mtNode}[4]{\node[textNodeStyle={text width=#2}] 
	(#1) at #3 {#4}}
\newcommand{\mtNodePos}[6]{\node[textNodeStyle={text width=#2},
	#3=of #4, #3=0pt, shift={#5}] (#1) {#6}}

% 添加文本框
\newcommand{\btNode}[4]{\node[textNodeStyle={text width=#2,%
		rectangle, draw, inner sep=3pt, boxStyle}] 
	(#1) at #3 {#4}}
\newcommand{\btNodePos}[6]{\node[textNodeStyle={text width=#2,%
		rectangle, draw, inner sep=3pt, boxStyle},%
		#3=of #4, #3=0pt, shift={#5}] (#1) {#6}}
	
\begin{tikzpicture}
\coordinate (origin) at (0, 0);
% single line text
\tNode{x}{(origin)}{Origin};
\tNodePos{x1}{right}{x}{(1, 0)}{Right};

% multiline text
\mtNode{a}{2cm}{([shift={(0, -1)}]origin)}{Origin Point\\ Another line};
\mtNodePos{a1}{2cm}{right}{a}{(.5, 0)}{Right Point\\ Another line};

% boxed text node
\btNode{b}{2cm}{([shift={(0, -2.3)}]origin)}{Boxed Point\\ Another line};
\btNodePos{b1}{2cm}{right}{b}{(.5, 0)}{Right Point\\ Another line};
\end{tikzpicture}
\end{dispExample*}